\documentclass[twocolumn, fontsize=10pt]{article}
\usepackage[margin=0.70in]{geometry}
\usepackage{lipsum,mwe,abstract}
\usepackage[english]{babel} 
\usepackage{fancyhdr} % Custom headers and footers
\pagestyle{fancyplain} % Makes all pages in the document conform to the custom headers and footers
\fancyhead{} 
\fancyfoot[C]{\thepage} % Page numbering for right footer
\setlength\parindent{0pt} 
\usepackage{amsmath,amsfonts,amsthm} % Math packages
\usepackage{wrapfig}
\usepackage{graphicx}
\usepackage{float}
\usepackage{subcaption}
\usepackage{comment}
\usepackage{enumitem}
\usepackage{cuted}
\usepackage{sectsty} % Allows customizing section commands
\usepackage{xcolor} % For colored text
\usepackage{hyperref} % Package for links
\hypersetup{
    colorlinks=true,
    linkcolor=blue,
    filecolor=magenta,      
    urlcolor=blue,
    }
\allsectionsfont{\normalfont \normalsize \scshape} % Section names in small caps and normal fonts

\renewenvironment{abstract} % Change how the abstract look to remove margins
 {\small
  \begin{center}
  \bfseries Resumen \vspace{-.5em}\vspace{0pt}
  \end{center}
  \list{}{%
    \setlength{\leftmargin}{0mm}
    \setlength{\rightmargin}{\leftmargin}%
  }
  \item\relax}
 {\endlist}

 \newenvironment{englishabstract} % Abstract en inglés
 {\small
  \begin{center}
  \bfseries Abstract \vspace{-.5em}\vspace{0pt}
  \end{center}
  \list{}{%
    \setlength{\leftmargin}{0mm}
    \setlength{\rightmargin}{\leftmargin}%
  }
  \item\relax}
 {\endlist}
 
\makeatletter
\renewcommand{\maketitle}{\bgroup\setlength{\parindent}{0pt} % Change how the title looks like
\begin{flushleft}
  \begin{center}
    {\color{black} \Large Facultad de Matemática y Computación. Universidad de La Habana. \\
    \color{black} Sistemas de Recuperación de Información. \\ \vspace{10pt}}
    \href{https://github.com/DanielMPMatCom/SRI-Project.git}{Repositorio del Proyecto en GitHub} % Add GitHub link here
    \vspace{10pt}
  \end{center}
  \textbf{\@title}
  \@author \\ 
  \@date
\end{flushleft}\egroup
}
\makeatother

%% ------------------------------------------------------------------- 

\title{

\Large Hibridación de técnicas de Sistemas de Recomendación. Ventajas del enfoque probabilístico en comparación con la factorización matricial. Implementación del algoritmo Naive Bayes Collaborative Filtering y su expansión para realizar recomendaciones a grupos de usuarios.  \\
[10pt] 
}
\date{\today}
\author{Daniel Machado Pérez - daniel.machado.0206@gmail.com \\
Daniel Toledo Martínez - daniel020126@gmail.com \\
Osvaldo R. Moreno Prieto - osvaldo020213@gmail.com}

\begin{document}

\twocolumn[ \maketitle ]

% --------------- ABSTRACT
\begin{abstract}
    El presente trabajo explora la hibridación de 
    técnicas en sistemas de recomendación, centrándose 
    en las ventajas del enfoque probabilístico frente a 
    la factorización matricial en cuanto a filtrado colaborativo. Se destaca cómo el 
    enfoque probabilístico, al proporcionar una 
    representación explícita de las incertidumbres, 
    mejora la interpretabilidad y explicación de las 
    recomendaciones generadas. En particular, 
    se implementa el algoritmo \textit{Naive Bayes Collaborative Filtering} (NBCF), 
    que combina la simplicidad del modelo \textit{Naive Bayes} 
    con el poder del filtrado colaborativo, 
    permitiendo recomendaciones precisas y explicativas. 
    Además, se expande este algoritmo para adaptarse 
    a la recomendación a grupos de usuarios, 
    abordando un área clave en la personalización 
    colectiva de contenidos. Los resultados demuestran 
    que el enfoque probabilístico no solo ofrece una 
    alternativa robusta a la factorización matricial, 
    sino que también potencia la capacidad del sistema 
    para ofrecer recomendaciones personalizadas y 
    comprensibles, tanto a individuos como a grupos.
\end{abstract}

\begin{englishabstract}
    This work explores the hybridization of techniques 
    in recommendation systems, focusing on the 
    advantages of the probabilistic approach over 
    matrix factorization in collaborative filtering. 
    It highlights how the probabilistic approach, 
    by providing an explicit representation of 
    uncertainties, improves the interpretability and 
    explanation of the generated recommendations. 
    In particular, the Naive Bayes Collaborative 
    Filtering (NBCF) algorithm is implemented, 
    combining the simplicity of the Naive Bayes model 
    with the power of collaborative filtering, 
    allowing for precise and explanatory 
    recommendations. Additionally, this algorithm is 
    expanded to adapt to group recommendations, 
    addressing a key area in the collective 
    personalization of content. The results 
    demonstrate that the probabilistic approach not 
    only offers a robust alternative to matrix 
    factorization but also enhances the system's 
    ability to deliver personalized and understandable 
    recommendations to both individuals and groups.
\end{englishabstract}

% --------------- KEYWORDS
\noindent \textbf{Palabras Clave}: Sistemas de Recomendación (RS), Filtrado Colaborativo (CF), Enfoque Probabilístico, Factorización Matricial, \textit{Naive Bayes Collaborative Filtering} (NBCF), \textit{Naive Pooling} (NBP).

\rule{\linewidth}{0.5pt}

% --------------- MAIN CONTENT

\section{Introducción}

\subsection{Descripción del tema y Técnicas de Recomendación}

Los sistemas de recomendación se han consolidado como 
herramientas esenciales en la personalización de 
contenidos en diversas plataformas digitales, 
desde servicios de \textit{streaming} hasta comercio electrónico. 
Estos sistemas tienen como objetivo filtrar grandes 
volúmenes de información y presentar a los usuarios 
elementos relevantes según sus preferencias. 
Entre las técnicas de recomendación más utilizadas, 
se destacan el filtrado colaborativo, el filtrado 
basado en contenido, el filtrado demográfico y los enfoques híbridos que 
combinan estos métodos.\\

El filtrado colaborativo, en particular, 
ha sido ampliamente adoptado debido a su capacidad 
para identificar patrones de comportamiento entre 
usuarios y ofrecer recomendaciones basadas en 
similitudes en sus interacciones previas. 
Este enfoque se puede implementar mediante técnicas 
basadas en memoria, que utilizan directamente las 
interacciones pasadas de los usuarios, o mediante 
técnicas basadas en modelos, que crean representaciones 
abstractas de las relaciones entre usuarios e ítems.

\subsection{Enfoques de Filtrado Colaborativo Basado en Modelos}

Dentro del filtrado colaborativo basado en modelos, 
dos enfoques destacan por su eficacia y popularidad: 
la factorización matricial y los modelos probabilísticos. 
La factorización matricial, como lo demuestra el 
algoritmo de descomposición en valores singulares (SVD), 
es una técnica poderosa para descomponer la matriz de 
interacciones usuario-ítem en factores latentes, 
permitiendo predicciones precisas de las preferencias 
de los usuarios. No obstante, su principal limitación 
radica en la falta de interpretabilidad de los factores 
latentes, lo que dificulta la explicación de las 
recomendaciones generadas.\\

En contraste, los modelos probabilísticos, 
como el \textit{Naive Bayes Collaborative Filtering} (NBCF), 
ofrecen una alternativa que, si bien puede alcanzar 
niveles de precisión similares a los de la 
factorización matricial, presenta la ventaja adicional 
de proporcionar interpretaciones más claras de las 
recomendaciones. El enfoque probabilístico permite 
modelar explícitamente la incertidumbre en las 
preferencias de los usuarios, lo que facilita la 
explicación del porqué de cada recomendación.

\subsection{Antecedentes y Justificación}

La elección del enfoque probabilístico como base de 
esta investigación se sustenta en los hallazgos 
presentados en la tesis doctoral titulada 
"Sistema recomendador híbrido basado en modelos 
probabilísticos". Esta tesis profundiza en las ventajas 
de utilizar modelos probabilísticos en sistemas de 
recomendación, destacando su capacidad para superar 
las limitaciones de los enfoques tradicionales de 
factorización matricial. Además, se presenta una 
implementación del algoritmo NBCF, que ha mostrado 
resultados prometedores en términos de precisión y 
explicabilidad.\\

Sin embargo, un área poco explorada en esta tesis es 
la capacidad de estos modelos para realizar 
recomendaciones a grupos de usuarios, una 
característica esencial en contextos como la 
recomendación de contenido para familias, grupos de 
amigos o equipos de trabajo. Esta investigación se 
propone expandir el algoritmo NBCF, siguiendo las 
recomendaciones de la tesis doctoral, para adaptarlo 
a la recomendación grupal, un desafío significativo 
en la personalización colectiva.\\

Para extender el algoritmo NBCF a la recomendación de 
grupos, se adoptará la idea del \textit{Naive Pooling} (NBP) 
propuesta en el artículo \textit{"Extended Naïve Bayes for 
Group Based Classification"}\cite{nbp}. El método NBP utiliza 
las probabilidades calculadas (en nuestro caso las ya 
con NBCF) y las 
combina de manera que se maximice la probabilidad 
conjunta para un grupo de usuarios, permitiendo así 
una clasificación efectiva de grupos con una alta 
coherencia en las recomendaciones. Este enfoque se 
considera particularmente adecuado para garantizar 
que todos los miembros del grupo reciban 
recomendaciones que reflejen tanto las preferencias 
individuales como las del colectivo.

\subsection{Dataset Seleccionado}

Para la evaluación de la implementación y expansión 
del algoritmo NBCF, se ha seleccionado el dataset 
FilmTrust\cite{filmtrust}, un conjunto de datos ampliamente utilizado 
en la investigación de sistemas de recomendación. 
FilmTrust contiene miles de calificaciones de 
películas proporcionadas por usuarios, lo que lo 
convierte en un recurso valioso para el análisis y 
desarrollo de modelos de recomendación. 
La riqueza y diversidad del dataset permiten probar 
la eficacia de los algoritmos en un entorno 
cercano a escenarios del mundo real. 
Este dataset fue uno de los utilizados en la tesis antes mencionada.

\subsection{Estructura del Trabajo}

El presente informe se estructura en varias secciones 
que desarrollan en detalle los diferentes aspectos de 
la investigación:

\begin{itemize} 
    \item \textbf{Estado del Arte}: Se revisa la literatura existente sobre técnicas de recomendación, con un enfoque en el filtrado colaborativo basado en modelos. 
    \item \textbf{Algoritmo NBCF}: Se describe la implementación del algoritmo NBCF y su funcionamiento. 
    \item \textbf{Expansión del Algoritmo NBCF}: Se presenta la adaptación del NBCF para realizar recomendaciones a grupos de usuarios, detallando las modificaciones realizadas. 
    \item \textbf{Evaluación de los Resultados}: Se analizan los resultados obtenidos tras la implementación y se comparan con enfoques tradicionales. 
    \item \textbf{Conclusiones}: Se resumen los hallazgos más relevantes de la investigación y se sugieren posibles direcciones futuras. 
\end{itemize}

\section{Estado del Arte}

Los sistemas de recomendación se han vuelto 
indispensables en la era de la información, 
donde los usuarios requieren herramientas que les 
permitan descubrir contenidos relevantes de manera 
eficiente. Existen diversas técnicas para abordar 
este problema, cada una con sus propias ventajas y 
limitaciones. Entre las más destacadas están el 
filtrado colaborativo, el filtrado basado en contenido, 
el filtrado demográfico y los enfoques híbridos. 
En esta sección, se revisarán las principales técnicas 
de recomendación, con un enfoque particular en el 
filtrado colaborativo y sus variantes basadas en 
modelos probabilísticos.

\subsection{Técnicas de Recomendación}

\begin{itemize}
    \item \textbf{Filtrado Colaborativo}: 
    Este enfoque se basa en la idea de que los 
    usuarios que han compartido preferencias similares 
    en el pasado probablemente coincidan en sus 
    elecciones futuras. El filtrado colaborativo puede 
    implementarse a través de dos métodos: 
    basado en memoria y basado en modelos. 
    Los enfoques basados en memoria, como el algoritmo 
    de k vecinos más cercanos (k-NN), utilizan 
    directamente la matriz de interacciones 
    usuario-ítem para realizar recomendaciones. 
    Por otro lado, los enfoques basados en modelos, 
    que incluyen técnicas como la factorización 
    matricial y los modelos probabilísticos, 
    construyen un modelo predictivo a partir de los 
    datos disponibles, ofreciendo recomendaciones más 
    precisas y escalables. \cite{estado_arte_sistemas_recomendacion}
    \item \textbf{Filtrado Basado en Contenido}: 
    Este método recomienda ítems a un usuario en 
    función de la similitud entre los ítems que ha 
    consumido previamente y otros ítems disponibles. 
    A diferencia del filtrado colaborativo, se basa en 
    las características de los ítems, como el género, 
    el director o los actores en el caso de películas. \cite{estado_arte_sistemas_recomendacion}
    \item \textbf{Filtrado Demográfico}: 
    Aunque menos utilizado en comparación con los 
    métodos anteriores, el filtrado demográfico se 
    basa en las características personales de los 
    usuarios, tales como su edad, género o ubicación. 
    Este enfoque supone que usuarios con 
    características demográficas similares tienden a 
    compartir preferencias similares. 
    Si bien puede ser útil para ciertos contextos, 
    su efectividad suele ser menor, ya que no tiene 
    en cuenta las interacciones individuales entre 
    usuarios e ítems. \cite{estado_arte_sistemas_recomendacion}
    \item \textbf{Enfoques Híbridos}:
    Estos combinan dos o más de las técnicas 
    mencionadas para mejorar la precisión y superar 
    las limitaciones inherentes a cada uno de los 
    métodos. Por ejemplo, un sistema híbrido puede 
    combinar el filtrado colaborativo con el filtrado 
    basado en contenido para ofrecer recomendaciones 
    más completas, tanto en precisión como en 
    diversidad. \cite{estado_arte_sistemas_recomendacion}
\end{itemize}

 
\subsection{Filtrado Colaborativo Basado en Modelos}
 
El filtrado colaborativo basado en modelos ha 
demostrado ser especialmente eficaz en sistemas de 
recomendación a gran escala. Entre los enfoques más 
destacados se encuentran la factorización matricial y
los modelos probabilísticos.

\begin{itemize}
    \item \textbf{Factorización Matricial}:
    Esta técnica ha demostrado ser una de las más 
    efectivas para el filtrado colaborativo. 
    En la factorización matricial, la matriz de 
    interacciones usuario-ítem se descompone en dos 
    matrices de menor dimensión que representan 
    factores latentes tanto para los usuarios como 
    para los ítems. Estos factores latentes permiten 
    realizar predicciones sobre las preferencias de 
    los usuarios al capturar características no 
    observadas explícitamente. Aunque la factorización 
    matricial, especialmente con algoritmos como la 
    descomposición en valores singulares (SVD), 
    ha demostrado ser muy precisa, su principal 
    limitación radica en la falta de interpretabilidad. 
    Los factores latentes no siempre son comprensibles
    o intuitivos para los usuarios, lo que dificulta
    la explicación de las recomendaciones. \cite{tesis_sistema_recomendador_hibrido}
    \item \textbf{Modelos Probabilísticos}:
    En contraste con la factorización matricial, 
    los modelos probabilísticos proporcionan una 
    representación más clara de las incertidumbres 
    en las preferencias de los usuarios. Uno de los 
    enfoques más representativos es el \textit{Naive Bayes 
    Collaborative Filtering} (NBCF), que combina la 
    simplicidad del modelo de \textit{Naive Bayes} con la 
    estructura del filtrado colaborativo. Este 
    enfoque permite una mayor interpretabilidad, 
    ya que ofrece una explicación probabilística de 
    las recomendaciones. Además, el NBCF ha mostrado 
    ser altamente adaptable a diferentes escenarios, 
    permitiendo la incorporación de nuevas variables 
    sin comprometer su eficiencia. \cite{tesis_sistema_recomendador_hibrido}

\end{itemize}

\subsection{Desarrollo en las Tesis y Papers} 

El enfoque probabilístico ha sido objeto de un 
estudio detallado en la tesis doctoral titulada 
"Sistema recomendador híbrido basado en modelos 
probabilísticos"\cite{tesis_sistema_recomendador_hibrido}. En esta tesis, se aborda la 
integración de modelos probabilísticos dentro de 
sistemas de recomendación híbridos, destacando cómo 
estos modelos no solo permiten una mayor precisión, 
sino que también aportan una capa de interpretabilidad 
que los métodos de factorización matricial no ofrecen. 
El autor propone un enfoque híbrido que combina los 
beneficios del filtrado colaborativo basado en modelos 
probabilísticos con técnicas de filtrado basado en 
contenido.

En el paper titulado \textit{"Extended Naïve 
Bayes for Group Based Classification"}\cite{nbp}, los autores 
presentan una extensión del clásico modelo 
\textit{Naive Bayes}, adaptándolo para su aplicación 
en la clasificación basada en grupos. Uno de los 
enfoques tratados, 
denominado \textit{Naive Pooling} (NBP), se centra en la 
agregación de probabilidades individuales para 
generar una probabilidad conjunta que permita la 
clasificación efectiva de grupos de usuarios. La 
metodología propuesta combina las probabilidades 
individuales de cada miembro del grupo para maximizar 
la coherencia y relevancia de la clasificación final. 
Este método resulta particularmente útil en contextos 
donde se deben generar recomendaciones o decisiones 
que reflejen tanto las preferencias individuales 
como la dinámica grupal. La capacidad de NBP para 
mantener la simplicidad del modelo Naive Bayes, 
al tiempo que amplía su aplicabilidad a escenarios 
grupales, lo convierte en una herramienta poderosa 
para la personalización colectiva en sistemas de 
recomendación.

Además, en el paper \textit{"A Collaborative Filtering 
Approach Based on Naive Bayes Classifier"}\cite{nbcf}, 
se profundiza en la implementación del NBCF y se 
demuestra su viabilidad como alternativa a los 
métodos tradicionales de filtrado colaborativo. 
Los resultados obtenidos en este estudio muestran 
que el NBCF puede igualar o superar el rendimiento de 
la factorización matricial, especialmente en datasets 
donde la interpretabilidad es tan importante como la 
precisión.

Finalmente, el trabajo \textit{"Hybrid Collaborative Filtering 
Based on Users' Rating Behavior"}\cite{hybrid_collaborative_filtering} presenta un enfoque 
híbrido que integra el comportamiento de valoración 
de los usuarios con el filtrado colaborativo. 
Este enfoque tiene una relevancia particular para 
nuestro proyecto, ya que permite ajustar las 
recomendaciones no solo en función de las 
interacciones pasadas, sino también considerando la 
manera en que los usuarios valoran los ítems, lo que 
aporta una capa adicional de personalización.

\section{Algoritmo NBCF}

\subsection{Introducción al Algoritmo NBCF}

El algoritmo \textit{Naive Bayes Collaborative 
Filtering} (NBCF) es una técnica innovadora dentro 
del campo de los sistemas de recomendación 
colaborativos. A diferencia de otros enfoques, 
como la factorización matricial, el NBCF aprovecha la 
simplicidad y efectividad del clasificador \textit{Naive Bayes} 
para predecir las preferencias de los usuarios en 
función de sus interacciones anteriores con ítems. 
Este método considera la probabilidad de que un 
usuario asigne una cierta calificación a un ítem, 
basándose en las calificaciones previas tanto del 
usuario como de otros usuarios con comportamientos 
similares.

\subsection{Formulación Matemática del Algoritmo NBCF}

El algoritmo NBCF se basa en la combinación de dos 
enfoques principales: basado en usuarios y basado 
en ítems. En cada uno de estos enfoques, se calcula la 
probabilidad a priori de que un usuario califique un 
ítem con un valor específico, y posteriormente se 
calcula el \textit{likelihood} para ajustar esta 
probabilidad en función de las calificaciones 
observadas.

\begin{itemize}
    \item \textbf{Enfoque basado en el usuario}:
    la probabilidad a priori y el \textit{likelihood} se
    calculan de acuerdo con los ítems que cada usuario ha votado. \cite{tesis_sistema_recomendador_hibrido}
    \item \textbf{Enfoque basado en ítems}:
    la probabilidad a priori y el \textit{likelihood} se calculan
    de acuerdo con los votos que cada ítem ha recibido. \cite{tesis_sistema_recomendador_hibrido}
    \item \textbf{Enfoque híbrido}:
    integra los enfoques basados en el usuario e ítems, a fin
    de complementarse uno con otro y mejorar la precisión del modelo. \cite{tesis_sistema_recomendador_hibrido}

\end{itemize}

Para el desarrollo de cada uno de estos enfoques se 
utiliza los siguientes conceptos de probabilidades:

\begin{itemize}
    \item \textbf{Probabilidad A Priori}: 
    En el enfoque basado en ítems, se calcula la 
    probabilidad a priori de que un usuario $u$ asigne 
    una calificación $y$ a un ítem $i$, 
    denotado como $P(r_u = y)$. De manera análoga, 
    en el enfoque basado en usuarios, se calcula la 
    probabilidad de que un ítem $i$ reciba una 
    calificación $y$ de cualquier usuario $u$, 
    denotado como $P (r_i = y)$.\\

    \begin{multline}
        P(r_i = y) = \frac{|\{u \in U|r_{u,i} = y\}| + \alpha}{|\{u \in U|r_{u,i} \neq \bullet\}| + |R|*\alpha}
    \end{multline} \cite{tesis_sistema_recomendador_hibrido}

    Donde:
    \begin{itemize}
        \item $U$ es el conjunto de usuarios.
        \item $r_{u,i}$ es la calificación otorgada por el usuario $u$ al ítem $i$.
        \item $\alpha$ es un parámetro para evitar 0 probabilidades.
        \item $|R|$ representa el número de votos plausibles.
        \item $\bullet$ representa la ausencia de voto.
    \end{itemize} 

    
    \item \textbf{Likelihood}: 
    El \textit{likelihood} ajusta la probabilidad a 
    priori mediante la consideración de la 
    información adicional disponible en las 
    calificaciones observadas. Para el enfoque basado 
    en ítems, esto se expresa como 
    $P(r_v = k | r_u = y)$, que representa la 
    probabilidad de que otro usuario $v$ califique 
    con $k$ un ítem que ha sido calificado con $y$ 
    por el usuario $u$. Similarmente, para el enfoque 
    basado en usuarios, se calcula el \textit{likelihood} 
    correspondiente $P(r_j = k|r_i = y)$.

    \begin{multline}
        P(r_j = k|r_i = y) =  \\ \frac{|\{u \in U|r_{u,j} = k  \land  r_{u,i} = y\}|+ \alpha}{|\{u \in U|r_{u,j} \neq \bullet \land r_{u,i} = y|\} + |R|*\alpha}
    \end{multline} \cite{tesis_sistema_recomendador_hibrido}


    
    \item \textbf{Combinación de Enfoques}: 
    En el enfoque híbrido, se integran las 
    probabilidades obtenidas de los enfoques basados 
    en usuarios y en ítems, proporcionando un modelo 
    más robusto y preciso para la predicción de 
    calificaciones.

    $P(r_{u,i} = y)$ representa el valor de
    probabilidad de que el usuario $u$ vote el ítem $i$ 
    con el voto $y$:

    \begin{multline}
        P(r_{u,i} = y) \propto \\ \left( P(r_u = y) \cdot \prod_{v \in Ui} P(r_v = r_{v,i} | r_u = y) \right)^{\frac{1}{1+|Ui|}} \\
        \cdot \left( P(r_i = y) \cdot \prod_{j \in Iu} P(r_j = r_{u,j} | r_i = y) \right)^{\frac{1}{1+|Iu|}}
    \end{multline} \cite{tesis_sistema_recomendador_hibrido}

    Donde:
    \begin{itemize}
        \item $I_u = \{i \in I \mid r_{u,i} \neq \bullet \}$ es el conjunto de ítems votados por el usuario $u$,
        \item y $U_i = \{u \in U \mid r_{u,i} \neq \bullet \}$ es el conjunto de usuarios que han votado el ítem $i$.
    \end{itemize} 
\end{itemize}
 

\subsection{Algoritmo NBCF: Implementación Paso a Paso}

El algoritmo NBCF se implementa de manera iterativa, 
asegurando la eficiencia computacional mediante 
técnicas de memorización que permiten evitar el 
recálculo innecesario de probabilidades. 
A continuación se describen los pasos del algoritmo:

\begin{itemize}
    \item \textbf{Inicialización}: 
    Se inicializan las probabilidades a priori y los 
    contadores utilizados en el cálculo de 
    \textit{likelihoods}.
    \item \textbf{Iteración sobre Usuarios e Ítems}: 
    Para cada usuario, se calcula la probabilidad de 
    cada calificación posible basada en las 
    calificaciones observadas para los ítems que 
    ha evaluado. De manera similar, se calcula para 
    cada ítem la probabilidad de recibir una 
    calificación específica basada en las 
    calificaciones anteriores recibidas.
    \item \textbf{Almacenamiento de Resultados}: 
    Los valores calculados se almacenan para ser 
    utilizados posteriormente en la predicción de 
    nuevas calificaciones, evitando la necesidad de 
    recalcular durante la fase de predicción.
\end{itemize}

Este enfoque garantiza que el NBCF no solo sea 
eficiente, sino que también se adapte bien a problemas 
de gran escala, manteniendo una complejidad 
computacional similar a la de otros métodos avanzados, 
como la factorización matricial \cite{tesis_sistema_recomendador_hibrido}.

\subsection{Resultados Experimentales y Comparativa}

El algoritmo NBCF ha demostrado su eficacia en 
múltiples conjuntos de datos públicos 
(MovieLens, FilmTrust, Yahoo, BookCrossing)\cite{tesis_sistema_recomendador_hibrido}, superando en 
varias métricas clave a los métodos de referencia más 
utilizados en el campo:
\begin{itemize}
    \item Error Medio Absoluto (MAE)
    \item Precisión y \textit{Recall}
    \item Ganancia acumulada descontada normalizada (nDCG)
\end{itemize}
Se compararon los siguientes enfoques:
\begin{itemize}
    \item NBCF (usuario)
    \item NBCF (ítem)
    \item NBCF (híbrido)
    \item BNMF
    \item GGM
    \item INBM
    \item Bi-CF
    \item NMF
\end{itemize}
Los resultados fueron los siguientes:
\begin{itemize}
    \item \textbf{MovieLens}: El enfoque 
    híbrido de NBCF ha mostrado mejoras significativas en 
    medidas de MAE, precisión y \textit{recall}, 
    así como el enfoque basado en ítems fue mejor en la 
    nDCG en comparación con enfoques 
    tradicionales.\cite{tesis_sistema_recomendador_hibrido}
    \item \textbf{FilTrust}: el MAE de NBCF (híbrido) 
    logra mejores resultados que los otros dos enfoques 
    propuestos, mientras que la precisión y recall son 
    mejores con NBCF (ítems) y NBCF (usuario). 
    Por otro lado, cuando aumenta el número de 
    recomendaciones, nDCG es mejor con el enfoque NBCF (híbrido).\cite{tesis_sistema_recomendador_hibrido}
    \item \textbf{Yahoo}: nDCG es mejor en NBCF
    (híbrido) en comparación con NBCF (ítem) y NBCF (usuario). 
    Además, la precisión y el \textit{recall} de los 
    tres enfoques propuestos presentan un resultado 
    casi similar entre ellos. Así mismo, hay una 
    superioridad lograda en MAE de NBCF (híbrido)
    con respecto a los otros enfoques propuestos.\cite{tesis_sistema_recomendador_hibrido}
    \item \textbf{BookCrossing}: NBCF (híbrido) y NBCF (ítem)
    proveen mejores resultados para nDCG en 
    comparación con los métodos de línea base de CF. 
    A diferencia de otros conjuntos de datos en 
    BookCrossing las métricas de precisión y \textit{recall} 
    son mejores para los métodos GGM, INBM y Bi-CF. 
    Sin embargo muestran una mejora con respecto a 
    los métodos BNMF y NMF. NBCF (híbrido) se muestra 
    superior al resto de los enfoques en cuanto al MAE.\cite{tesis_sistema_recomendador_hibrido}
\end{itemize}

\subsection{Conclusión}

El algoritmo NBCF representa una mejora significativa 
en el ámbito de los sistemas de recomendación 
colaborativos, combinando la simplicidad del 
clasificador \textit{Naive Bayes} con técnicas de filtrado 
colaborativo para ofrecer recomendaciones precisas y 
eficaces. Su capacidad para integrar múltiples 
enfoques y adaptarse a diferentes escenarios lo 
convierte en una herramienta valiosa para la mejora de 
la experiencia del usuario en plataformas de 
recomendación.

\section{Expansión del Algoritmo NBCF}

\subsection{Algoritmo NBP. Formulación Matemática}
En los sistemas de recomendación modernos, la 
capacidad de realizar recomendaciones efectivas 
no solo a usuarios individuales, sino también a grupos, 
se ha vuelto un aspecto crucial. Contextos como la 
recomendación de contenido para familias, equipos de 
trabajo o grupos de amigos demandan un enfoque que 
pueda considerar y equilibrar las preferencias 
individuales dentro de un colectivo, maximizando la 
satisfacción general del grupo. En respuesta a esta 
necesidad, surge el algoritmo NBP, una extensión del 
modelo clásico de \textit{Naive Bayes} adaptado 
específicamente para la clasificación y recomendación 
a grupos de usuarios.

El algoritmo se fundamenta en la misma premisa básica 
que el \textit{Naive Bayes}, es decir, la independencia 
condicional de las características dadas las clases. 
Sin embargo, lo que distingue a NBP es su capacidad 
para combinar las probabilidades individuales de los 
miembros de un grupo, produciendo una probabilidad 
conjunta que guía la recomendación o clasificación 
final para todo el grupo.

Matemáticamente, la probabilidad conjunta para un 
grupo $G$ de $n$ usuarios con respecto a una clase $C_l$ 
se puede expresar como:

\begin{multline}
    P(C_l) = \frac{\prod_{k=1}^{n} P(C_l|x_k)}{\prod_{k=1}^{n} P(c_l|x_k) + \prod_{k=1}^{n} (1-P(c_l|x_k))}
\end{multline} \cite{nbp}

Donde:
\begin{itemize}
    \item $x_k$ es un usuario del grupo $G$,
    \item $P(C_l|x_k)$ es la probabilidad de la 
    clase $C_l$ dada para el usuario individual $x_k$. 
\end{itemize} 

El algoritmo NBP maximiza esta probabilidad conjunta, 
seleccionando la clase $C_l$ que mayor valor obtenga, 
lo que se traduce en una recomendación para el grupo 
que refleje un consenso entre las preferencias 
individuales.

\subsection{Ventajas y Desventajas de NBP}

El algoritmo Naive Pooling presenta varias ventajas notables:
\begin{itemize}
    \item \textbf{Simplicidad y Eficiencia}: 
    Al estar basado en el modelo \textit{Naive Bayes}, 
    NBP hereda la simplicidad computacional y la 
    eficiencia del mismo, lo que permite su 
    implementación en sistemas a gran escala sin 
    requerir un costo computacional elevado.
    \item \textbf{Interpretabilidad}:
    Una de las fortalezas del enfoque probabilístico 
    es su capacidad para proporcionar una explicación 
    clara de las recomendaciones basadas en 
    probabilidades. Esto facilita la interpretación de 
    por qué un grupo recibió una recomendación 
    específica.
    \item \textbf{Adaptabilidad}:
    NBP es altamente adaptable y puede integrarse con 
    facilidad en sistemas de recomendación existentes 
    que ya utilicen el enfoque \textit{Naive Bayes} 
    para recomendaciones individuales. Esto permite a 
    los desarrolladores extender sus sistemas a la 
    recomendación grupal sin necesidad de una 
    reingeniería significativa.
\end{itemize}

A pesar de sus ventajas, el algoritmo NBP también 
presenta algunas limitaciones que deben ser consideradas:

\begin{itemize}
    \item \textbf{Suposición de Independencia}:
    Al igual que \textit{Naive Bayes}, NBP asume que 
    las preferencias de los usuarios dentro del grupo 
    son condicionalmente independientes, lo cual puede 
    no reflejar adecuadamente la realidad, donde las 
    preferencias de los usuarios pueden estar 
    correlacionadas.
    \item \textbf{Equidad en la Recomendación}:
    NBP no tiene en cuenta explícitamente la equidad 
    entre las preferencias individuales dentro del 
    grupo. Es decir, podría favorecer las preferencias 
    de algunos usuarios sobre otros, especialmente si 
    las probabilidades individuales de ciertos miembros 
    del grupo son mucho más altas que las de los demás.
    \item \textbf{Escalabilidad con Grupos Grandes}:
    Aunque eficiente, a medida que el tamaño del grupo 
    aumenta, la combinación de probabilidades puede 
    llevar a situaciones donde las recomendaciones 
    sean dominadas por usuarios con preferencias 
    extremas, lo que podría reducir la diversidad y la 
    satisfacción global del grupo.
\end{itemize}

\subsection{Aplicación del Algoritmo NBP en la Expansión de NBCF}

En el contexto de esta investigación, el algoritmo 
\textit{Naive Pooling} se integrará con el modelo 
\textit{Naive Bayes Collaborative Filtering} (NBCF). 
Teniendo previamente calculados los valores de las 
probabilidades para cada usuario con NBCF, ya sea con 
su enfoque basado en usuario, basado en ítems o híbrido,
se puede efectuar la fórmula de NBP y obtener el resultado 
que se quiere. De esta forma, se extiende 
la funcionalidad del NBCF para proporcionar 
recomendaciones no solo a individuos, sino también a 
grupos de usuarios, conservando la precisión y 
explicabilidad del enfoque probabilístico. 
Esta expansión permitirá que el sistema no solo 
mantenga la calidad en las recomendaciones 
individuales, sino que también pueda satisfacer las 
necesidades de colectivos en situaciones donde la 
personalización grupal es esencial.

\section{Evaluación de los resultados}

\section{Conclusiones}

\renewcommand\refname{Referencias}

\begin{thebibliography}{}

    \bibitem{nbcf}
    Valdiviezo-Diaz, P., Ortega, F., Cobos, E., \& Lara-Cabrera, R. (2019). A collaborative filtering approach based on Naïve Bayes classifier. IEEE Access, 7, 108581-108592.
    
    \bibitem{estado_arte_sistemas_recomendacion}
    González, O. E., \& Jacques, S. M. (2017). Estado del arte en los sistemas de recomendación. Res. Comput. Sci., 135, 25-40.
    
    \bibitem{hybrid_collaborative_filtering}
    Ortega, F., Rojo, D., Valdiviezo-Diaz, P., \& Raya, L. (2018). Hybrid collaborative filtering based on users rating behavior. IEEE Access, 6, 69582-69591.
    
    \bibitem{tesis_sistema_recomendador_hibrido}
    Valdiviezo, P. M. (2019). Sistema recomendador híbrido basado en modelos probabilísticos (Doctoral dissertation, Universidad Politécnica de Madrid).

    \bibitem{nbp}
    Samsudin, N. A., \& Bradley, A. P. (2014). Extended naïve bayes for group based classification. In Recent Advances on Soft Computing and Data Mining: Proceedings of The First International Conference on Soft Computing and Data Mining (SCDM-2014) Universiti Tun Hussein Onn Malaysia, Johor, MalaysiaJune 16th-18th, 2014 (pp. 497-505). Springer International Publishing.

    \bibitem{filmtrust}
    J. Golbeck, J. Hendler, “FilmTrust: movie recommendations using trust in web-based social networks”, CCNC 2006, 3rd IEEE Consumer Communications and Networking Conference, 2006, DOI: 10.1109/CCNC.2006.1593032 
    
\end{thebibliography}
    
    


\end{document}
